\documentclass{beamer}

\usepackage[brazil]{babel}
\usepackage[utf8]{inputenc}
\usepackage[T1]{fontenc}

\usetheme{Madrid}
\setbeamertemplate{navigation symbols}{}

\title[Lógica Computacional]{Lógica Computacional}

\author[Diego S. C. Nascimento]{Diego Silveira Costa Nascimento}

\institute[IFRN]{
Instituto Federal de Educação, Ciência e Tecnologia do Rio Grande do Norte\\
diego.nascimento@ifrn.edu.br
}

\date[\today]{\today}

\begin{document}

\begin{frame}[plain]
	\includegraphics[scale=0.2]{img/IFRN}
	\titlepage
\end{frame}

\logo{\includegraphics[scale=0.1]{img/IFRN}}

\begin{frame}
	\frametitle{Ementa do Curso}
  	\tableofcontents
\end{frame}

\AtBeginSection[]{
	\begin{frame}
		\frametitle{Ementa do Curso}
		\tableofcontents[currentsection]
	\end{frame}
}

\section{Introdução}

\begin{frame}
	\frametitle{Lógica}

	\begin{block}{Definição}
		É a ciência das leis ideais do pensamento e a arte de aplicá-las à pesquisa e à demonstração da verdade.
	\end{block}\vfill
	
	\begin{itemize}
		\item Deriva do Grego (logos); e
		\item Significa:
			\begin{itemize}
			\item palavra;
			\item pensamento;
			\item ideia;
			\item argumento;
			\item relato;
			\item razão
			\item lógica; ou
			\item princípio lógico.
			\end{itemize}
	\end{itemize}
\end{frame}

\begin{frame}
\frametitle{Origem}

\begin{itemize}
	\item A Lógica teve início na Grécia em 342 a.C.;
	\item Aristóteles sistematizou os conhecimentos existentes em Lógica, elevando-a à categoria de ciência;
	\item Obra chamada Organon (Ferramenta para o correto pensar);
	\item Aristóteles preocupava-se com as formas de raciocínio que, a partir de conhecimentos considerados verdadeiros, permitiam obter novos conhecimentos; e
	\item A partir dos conhecimentos tidos como verdadeiros, caberia à Lógica a formulação de leis gerais de encadeamentos lógicos que levariam à descoberta de novas verdades.
\end{itemize}\vfill

\begin{columns}[c] 
	\column{.3\textwidth}
	\begin{exampleblock}{Aristóteles}
		\center
		\includegraphics[scale=0.18]{img/aristoteles}
	\end{exampleblock}
	
	\column{.3\textwidth}
	\begin{exampleblock}{Organon}
		\center
		\includegraphics[scale=0.1]{img/organon}	
	\end{exampleblock}
	
\end{columns}
\end{frame}

\begin{frame}
\frametitle{Princípios Lógico}

A Lógica Formal repousa sobre três princípios fundamentais que permitem todo seu desenvolvimento posterior, e que dão validade a todos os atos do
pensamento e do raciocínio.

\begin{exampleblock}{Princípio da Identidade}
Afirma $A = A$ e não pode ser $B$, o que é, é.
\end{exampleblock}\vfill
	
\begin{exampleblock}{Princípio da Não Contradição}
$A = A$ e nunca pode ser não-$A$, o que é, é e não pode ser sua negação, ou seja, o ser é, o não ser não é.
\end{exampleblock}\vfill

\begin{exampleblock}{Princípio do Terceiro Excluído}
Afirma que Ou $A$ é $x$ ou $A$ é $y$, não existe uma terceira possibilidade.
\end{exampleblock}

\end{frame}

\section{Lógica Proposicional}

\begin{frame}
\frametitle{Proposição}

\begin{itemize}
	\item Chama-se proposição todo o conjunto de palavras ou símbolos que exprimem um pensamento de sentido completo;
	\item As proposições transmitem pensamentos; e
	\item Afirmam fatos ou exprimem juízos que formamos a respeito de determinados entes.
\end{itemize}\vfill

\begin{exampleblock}{Exemplos}
A Lua é um satélite da terra\\
Sócrates é um homem\\
Eu estudo lógica\\
Não está chovendo
\end{exampleblock}
\end{frame}

\begin{frame}
\frametitle{A Linguagem}

\begin{block}{Considere o conjunto de símbolos:}
$ A = \{(,), \neg, \wedge, \vee, \rightarrow, \leftrightarrow, p, q, r, ... \}$
\end{block}\vfill

\begin{itemize}
	\item A esse conjunto é chamado de alfabeto da Lógica Proposicional;
	\item As letras são símbolos não lógico (letras sentenciais); e
	\item O restante são símbolos lógicos (parênteses e conectivos lógicos).
\end{itemize}
\end{frame}

\begin{frame}
\frametitle{Letras Sentenciais}

As letras sentenciais são usadas para representar proposições elementares ou atômicas, isto é, proposições que não possuem partes que sejam também proposições.
\vfill
\begin{exampleblock}{Exemplos}
p = O céu é azul\\
Q = Eu estudo lógica\\
r = 2 + 2 = 4\\
s = Sócrates é um homem
\end{exampleblock} \vfill

\begin{alertblock}{Importante}
As partes dessas proposições não são proposições mais simples, mas sim, componentes subsentenciais: expressões, palavras, sílabas ou letras.
\end{alertblock}
\end{frame}

\begin{frame}
\frametitle{Conectivos Lógicos}

\begin{itemize}
	\item As proposições compostas são obtidas combinando proposições simples através de certos termos chamados conectivos;
	\item A Lógica dispõe de cinco tipos de conectivos e seus operadores:
	\begin{itemize}
		\item Não (Negação), \structure{$\neg$};
		\item E (Conjunção), \structure{$\wedge$};
		\item Ou (Disjunção), \structure{$\vee$};
		\item Se -- então (Condicional), \structure{$\rightarrow$};e
		\item Se e somente se (Bicondicional), \structure{$\leftrightarrow$}.
	\end{itemize}
\end{itemize} \vfill

\begin{exampleblock}{Exemplos}
\structure{Não} está chovendo\\
Está chovendo \structure{e} está ventando\\
Está chovendo \structure{ou} está nublado\\
\structure{Se} choveu, \structure{então} está molhado\\
Aprenderá \structure{se e somente se} estudar
\end{exampleblock}
\end{frame}

\begin{frame}
\frametitle{Operador de Negação: $\neg$}

A característica peculiar da negação, tal como ela se apresenta na lógica proposicional clássica, é que toda proposição submetida à operação de negação resulta na sua contraditória.
\vfill

\begin{exampleblock}{Exemplos}
p = Está chovendo.\\
Ler-se $\neg p$, como: ``Não está chovendo''.
\end{exampleblock}\vfill

\begin{alertblock}{Importante}
O fato expresso por uma proposição não pode ocorrer ao mesmo tempo e sob o mesmo modo e circunstância que o fato expresso pela negação dessa mesma proposição.
\end{alertblock}
\end{frame}

\end{document}